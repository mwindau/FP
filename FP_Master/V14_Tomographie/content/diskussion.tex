\section{Diskussion}
\label{sec:Diskussion}

Die Vermessung des zweiten W\"urfels ergibt f\"ur den Absorptionskoeffizienten eine sehr gute \"Ubereinstimmung mit Eisen.
Der ermittelte Absorptionskoeffizient des dritten W\"urfels hat einen sehr geringen Wert der auf ein leichtes Material vermuten l\"asst.
Das Gewicht des W\"urfels w\"ahrend der Messung, schlie{\ss}t dies jedoch aus und passt eher zu einem Metall. 

Ein Grund f\"ur die hohe Abweichung des Absorptionskoeffizienten vom dritten W\"urfel zu seinem wahren Theoriewert, liegt in dem Strahlenweg durch die geschweißten Ecken des W\"urfels. Zus\"atzlich
existiert der lange Strahlenweg au{\ss}erhalb des
W\"urfels, da die Reduzierung der Intensit\"at durch Luft nicht weiter betrachtet wird. Die daraus folgende Abweichung wird jedoch als nicht sehr hoch eingestuft.
Andere Gr\"unde liegen in der Aluminiumh\"ulle der W\"urfel und der Ausdehnung des Strahlenganges.
W\"ahrend versucht wird dem Effekt der H\"ulle durch Anpassung der Eingangsintensit\"at entgegenzuwirken, kann die Ausdehnung des Strahles und damit das durchdringen von benachbarten Elementarw\"urfeln nicht verhindert werden. 
Zuletzt entstehen Abweichungen durch die Ausrichtung der W\"urfel im Strahl, die per Augenma{\ss} durchgef\"uhrt wurde.
Dies f\"uhrt zum Teil zu falschen Projektionsachsen, die eine andere Dicke des Materials durchqueren als angenommen.