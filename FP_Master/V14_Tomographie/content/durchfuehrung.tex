\section{Durchführung}
\label{sec:Durchführung}

        Die Messdauer beträgt bei jeder Einzelmessung 300\,s, um nach der Poissonverteilung 
        sicher eine statistische Unsicherheit von unter 3\,\% zu erhalten. 
        Zunächst wird eine Leermessung ohne Würfel durchgeführt und der Inhalt des 
        Vollenergiepeaks bestimmt. Danach werden der erste, zweite und dritte Würfel 
        für verschiedene Projektionen vermessen. Hier werden allerdings nur die Projektionen 
        2, 5 und 9 aufgenommen (siehe Abb. \ref{fig:wuerfel}), da die Würfel homogen sind 
        und somit theoretisch bereits eine Messung zur Bestimmung des Materials ausreichend wäre.
        Für den letzten Würfel, der aus 9 Elementarwürfeln besteht, wird der Inhalt des 
        Vollenergiepeaks für alle in Abb. \ref{fig:wuerfel} darstellten Projektionen bestimmt.