\section{Durchführung}
\label{sec:Durchführung}


\subsection{Invertierender-Linearverstärker}

Zunächst wird die Schaltung des Linearverstärkers aus Abbildung \ref{fig:linear} auf einem Steckbrett aufgebaut.
Es wird für 4 verschiedene Widerstandskombinationen der Verstärkungsfaktor und die Phase in Abhängigkeit der
Frequenz aufgezeichnet.

\subsection{Integrator und Differentiator}

Als nächstes wird die Schaltung aus Abbildung \ref{fig:integrator} des Integrators und Differentiators (Abbildung \ref{fig:diffdiff}) aufgebaut.
Es gilt für eine Sinusspannung den Verstärkungsfaktor in Abhängigkeit von der Frequenz zu messen, um
die gewählte Zeitkonstante zu überprüfen.
Anschließend wird die Integration- beziehungsweise Differentiationsfunktion der Schaltungen für
Sinus-, Dreieck- und Rechteckspannungen verifiziert.

\subsection{Schmitt-Trigger}

Es wird die Schaltung aus Abbildung \ref{fig:schmittimacschmitt} für einen Schmitt-Trigger aufgebaut und für eine Sinusspannung
langsam die Amplitude solange erhöht, bis das Signal anfängt zu kippen. Die Hysteresekurve wird
auf einem Oszilloskop sichtbar gemacht und abgespeichert. 

\subsection{Signalgenerator}

Die Schaltung wird zu der Schaltung aus Abbildung \ref{fig:generator} für einen Signalgenerator ergänzt.
Es wird mit einer Sinusspannung als Eingangsssignal eine Dreiecksspannung erzeugt und deren
Amplitude und Frequenz gemessen.

\subsection{Variierende Amplitude}

Zuletzt wird die Schaltung aus Abbildung \ref{fig:varamp} für die variierende Amplitude aufgebaut. Es wird eine
gedämpfte Schwingung im Schwingkreis erzeugt, indem eine Rechteckspannung auf die Schaltung
mit negativem $\eta$ gegeben wird. Anschließend wird eine Schwingung für positives $\eta$
erzeugt. Es ergibt sich eine charakteristische Frequenz, welche gemessen wird.