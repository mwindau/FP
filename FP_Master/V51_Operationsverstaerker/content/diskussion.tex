\section{Diskussion}
\label{sec:Diskussion}

        In der Schaltung des invertierenden Linearverstärkers zeigen die Messwerte für alle Widerstandsverhältnisse 
        einen erwarteten Verlauf mit drei Teilen: Plateau, Übergang und Abfall. 
        Der Mittelwert des Bandbreitenprodukts liegt bei 748,75\,kHz. Für die Widerstandsverhältnisse 10, 33,1, 3,31 und 
        2,96 ergeben sich prozentuale Abweichungen von 36,6\,\%, 41,6\,\%, 0,6\,\% und 5,6\,\% vom Mittelwert.
        Die Phasenbeziehung zwischen Ein- und 
        Ausgangsspannung fällt mit steigender Frequenz tiefpassartig ab. Beim Umkehrintegrator und -differentiator 
        sind die erwarteten Ausgangsspannungsbilder für alle eingestellten Eingangsspannungen auf dem 
        Oszilloskop zu sehen. Die Proportionalität zwischen Frequenz und Ausgangsspannung ist auch für beide Schaltungen so,
        wie zuvor angenommen: Der Integrator liefert eine Spannung, die Antiproportional zur Frequenz ist, was
        aus dem Fitparameter $b=-0,923$ hervorgeht, welcher um 7,7\,\% vom erwarteten Parameter $b=-1$ abweicht.
        Der Differentiatior sollte eine zur Frequenz proportionale Spannung ausgeben mit $b=1$. Der 
        gemessene Parameter $b=0,972$ weicht um 2,8\,\% vom erwarteten ab.
        Beim nicht-invertierenden Schmitt-Trigger ist die beobachtete Schalthysterese 
        viel kleiner als die zuvor errechnete. Dies ist im Rahmen normaler Fehler nicht zu erklären, weshalb vermutlich 
        die Größenordnung eines Bauteils oder der Ausgangsspannung falsch notiert wurde. Die Frequenz 
        des Signalgenerators liegt in der gleichen Größenordnung wie die theoretisch berechnete. 
        Allerdings beträgt die Abweichung trotzdem 71\%. 
        Die letzte Messung stimmt nicht mit den Erwartungen überein, da hier ein sinusförmiger Verlauf 
        zu sehen sein sollte, allerdings nur eine Schwingung innerhalb der Plateaus des Dreiecks 
        zu erkennen ist. Dies spricht dafür, dass die Schaltung richtige Elemente enthält, welche 
        eine Schwingung erzeugen, aber dass auch irgendwo ein Fehler vorliegen muss, da der Rechtecksanteil 
        erhalten bleibt. 