\section{Theorie}
\label{sec:Theorie}

Zum Verstehen eines Sagnac-Interferometers wird zunächst die benötigte grundlegende Theorie des Lichts, der Interferenz und der Polarisation erl\"autert. Im Anschluss wird
auf die Bestimmung von Brechungsindizes in Glas und Luft eingeganen.

\subsection{Licht, Interferenz und Polarisation}

Es handelt sich bei Licht um elektromagnetische Wellen, weswegen sie im Falle von monochromatichem Licht g\"anzlich durch ein elektrisches Feld der Form
\begin{align}
    \vec{E} = \vec{E_0} \, \exp(i(\omega t - \vec{k}\vec{r}))
\end{align}
beschrieben werden k\"oennen. Dabei wird die Polarisation des Feldes durch $\vec{E_0}$ dargestellt.
Durch die \"Uberlagerung von zwei Wellen $\vec{E_1}$ und $\vec{E_2}$ k\"onnen Interferenzeffekte enstehen.
Eine Messbare Gr\"o\ss{}e f\"ur die \"Uberlagerung stellt dabei die Intensit\"at dar:
\begin{align}
    I_{\symup{Laser}} \approx \bigl< |\vec{E_1}+\vec{E_2}|^2 \bigr> = \vec{E_01}^2+\vec{E_02}^2 + 2\vec{E_01}\vec{E_02}\cos(\delta)\cos(\Delta \Phi)
\end{align}
Dabei steht $\delta$ f\"ur den Polarisationswinkel zwischen den beiden Feldervektoren und $\Delta \Phi$ f\"ur die Phasendifferenz zwischen den Feldern.
Falls die Wellen senkrecht zueinander polarisiert sind ($\delta=\pi/2$) tritt kein Interferenzeffekt auf.
Die Polarisation eines elektrischen Feldes kann mit einem Polarisationsfilter beeinflu\ss{}t werden, indem diese den Lichtanteil senkrecht zum eingestellten Filterwinkel rausfiltert.
Durchquert ein Strahl einen Polarisationsfilter mit einem Filterwinkel von $\theta$, l"asst sich die Polarisation im Faktor $\vec{E_0}$ durch
\begin{align}
    E_0 = (E_x\vec{e_x}=E_y\vec{e_y})\,\left(\frac{\cos\theta}{\sin\theta}\right)
\end{align}
beschreiben. Werden die elektrischen Felder auf eine gemeinsame Polarisation gebracht ($\delta=0$), ergibt sich insgesamt f\"ur die Intensit\"at:
\begin{align}
    I \approx I_{\symup{Laser}}(1+2\cos\theta\sin\theta\cos(\Delta\Phi))
\end{align}
Die Intensit\"at wird maximal f\"ur konstruktive Interferenz, also einer Phasendifferenz von $0,2\pi...$ und minimal f\"ur destruktive Interferenz, also einer
Phasendifferenz von $\pi,3\pi...$:
\begin{align}
    I_{\symup{max/min}} &= I_{\symup{Laser}}(1\pm 2\cos\theta\sin\theta)
\end{align}
Es l\"asst sich f\"ur Interferometer ein Kontrast, das hei\ss{}t eine "Sichtbarkeit" der Interferenz, berechnen:
\begin{align}
    K &= \frac{I_{\symup{max}} - I_{\symup{min}}}{I_{\symup{max}} + I_{\symup{min}}}\\
      &= |\sin(2\theta)|
    \label{eq:kontrast}
\end{align}

\subsection{Bestimmung von Brechungsindizes}

Bei der Bestimmung der Brechungsindizes wird im Vorgehen zwischen Festk\"orpern und Gasen unterschieden.

Propagiert einer der Lichtstrahlen durch ein Glas der L\"ange $T$, ensteht eine Phasendifferenz zwischen den Strahlen.
Mittels dem Snelliusschen Brechunggesetz und einer Kleinwinkeln\"aherung kann die auftretende Phasendifferenz  wie
folgt dargestellt werden:
\begin{align}
    \Delta \Phi = \frac{2\pi}{\lambda_{\symup{vac}}}T\left(\frac{n-1}{2n}\theta^2+\symcal{O}(\theta^4)\right)
\end{align}
Dabei steht $\lambda_{\symup{vac}}$ f\"ur die Wellenl\"ange des Lichts im Vakuum. Mit der Anzahl $M$ der Interferenzextrema und einem Faktor von 2, da beide Lichtstrahlen durch die Probe propagieren,
l\"asst sich nun folgende Gleichung aufstellen:
\begin{align}
    M = \frac{2\Delta\Phi}{2\pi} \approx frac{T(n-1)}{2n\lambda_{\symup{vac}}}\theta^2
\end{align}
Da sich die Probe in beiden Lichtstrahlen mit einem jeweils anderen $\theta$ befindet, wird die Gleichung durch den linearen Term der Taylorreihe gen\"ahert:
\begin{align}
    M \approx \frac{T(n-1)}{n\lambda_{\symup{vac}}}(2\theta_0\Delta\theta)
\end{align}

Bei der Bestimmung des Brechungsindizes von Luft, f\"uhrt man eine Gaszelle in einen der Strahleng\"ange ein. Propagiert der Lichtstrahl nun durch die Gaszelle ensteht eine Phasendifferenz zwischen
den Strahlen, die unteranderem abh\"angig vom Unterschied zwischen
den Brechungsindizes der Luft in und au\ss{}erhalb der Gaszelle ist. Analog zum Vorgehen beim Glas, wird wieder die Anzahl $M$ der Interferenzextrema betrachtet:
\begin{align}
    M = \frac{\Delta\Phi}{2\pi} = \frac{\Delta nL}{\lambda_{\symup{vac}}}
\end{align}
Dabei steht $L$ f\"ur die L\"ange der Gaszelle.
Im Gegensatz zum Brechungsindex im Glas, ist der Brechungsindex im Gas stark vom Druck $p$ abh\"angig. Mit dem Lorentz-Lorenz-Gesetz l\"asst sich eine N\"aherung \cite{lorentz} f\"ur den
Brechungsindex vom Gas ermitteln:
\begin{align}
    n \approx \sqrt{1+\frac{3Ap}{RT}}
\end{align}