\section{Diskussion}
\label{sec:Diskussion}


            Der Kontrast des Interferometers ist bei 45° am besten, was mit den Vorüberlegungen übereinstimmt.
            Er liegt bei 0,818. Mit diesem Kontrast können gute Messungen durchgeführt werden. Der Wert ist jedoch
            vermutlich durch präziesere Ausrichtung noch vergrößerbar. Der Fit der Werte macht deutlich, 
            dass die Abhängigkeit des Kontrastes vom Polarisationswinkel den zuvor angenommenen Verlauf annimmt.

            Der Brechungsindex von Glas liegt bei etwa 1,5 \cite{rii}, weshalb der mit dem Interferometer ermittelte 
            Wert von 1,59 nahe an der Realität liegt. Der Brechungsindex von Luft liegt, genau wie der 
            durch das Interferometer gemessene, bei etwa 1 \cite{refr_index_air}, sodass hier eine gute Übereinstimmung
            zwischen Literatur und Praxis vorliegt. 
