\section{Durchführung}
\label{sec:Durchführung}



        \subsection{Vermessung des Magnetfeldes}

            Der Strom zur Speisung des Magnetfeldes wird auf sein Maximum eingestellt.
            Danach wird eine Hall-Sonde in den Magneten geführt. Sie kann parallel 
            zur Richtung des Lichts, welches durch den Magneten fällt, verschoben werden.
            Für verschiedene Positionen der Hallsonde um das maximale Magnetfeld herum wird 
            der Wert des Magnetfeldes notiert.

        \subsection{Justage}

            Die Justage wird ohne eingesetzte Probe und Interferenzfilter
            durchgeführt. Zunächst wird überprüft, ob der Strahl nach dem
            Analysatorprisma verschwindet, wenn am Polarisationsprisma gedreht wird.
            Danach muss die Position der \SI{100}{\milli\meter}-Linse so gesetzt werden,
            dass das Licht bei beiden Photowiderständen ankommt.
            Im Anschluss wird am Lichtzerhacker und am Selektivverstärker
            die gleiche Frequenz von \SI{450}{\hertz} eingestellt. Die Güte des 
            Selektivverstärkers wird für das Experiment maximiert.
            Durch Bewegung der Lichtquelle und der Kondensorlinse wird nun die 
            maximal mögliche Lichtintensität eingesetllt. Nach Einsetzen der 
            Probe und des Filters wird versucht, durch Drehung des Polaristionsprismas 
            ein Nullsignal nach dem Selektivverstärker zu erzeugen. Wenn dies funktioniert,
            können Messungen durchgeführt werden.

        \subsection{Untersuchung der GaAs-Proben}

            Es werden die gleichen Messungen an einer hochreinen und an zwei $n$-dotierten 
            GaAs-Proben durchgeführt, wobei für eine Probe 
            $n=\SI{1.2e18}{\per\cubic\centi\meter}$ 
            und für die andere $n=\SI{2.8e18}{\per\cubic\centi\meter}$ gilt.

            Die jeweilige Probe wird an ihren Platz im Magneten gesetzt. Danach wird einer 
            der neun Interferenzfilter angebracht. Nun wird das höchstmögliche positive Magnetfeld 
            eingestellt. Durch Verstellen des Goniometers am ersten Glan-Thompson-Prisma wird das 
            Differenzsignal minimiert. Der eingestellte Winkel wird notiert. Für das entsprechende
            negative Magnetfeld wird im Anschluss ebenfalls das Differenzsignal minimiert und 
            wieder der Winkel des Goniometers aufgenommen.
