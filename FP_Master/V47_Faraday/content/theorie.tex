\section{Theorie}
\label{sec:Theorie}

        \subsection{Effektive Masse}

            In der Effektiv-Massen-Näherung wird das Leitungsband eines Halbleiter-Kristalls 
            als parabelförmig mit einem Minimum bei $k=0$ angenommen. Das Energieband 
            für einen isotropen Kristall lässt sich durch die Gleichung 

            \begin{equation}
                \varepsilon(\vec{k}) = \varepsilon(0) + \frac{1}{2}\sum_{i=1}^3
                \left(\frac{\partial \epsilon^2}{\partial k_i^2}\right)_{k=0}\cdot k_i^2 + \,
                \mathcal{O}(k^3) = \varepsilon(0) + \frac{\hbar^2k^2}{2m^*} + \mathcal{O}(k^3)
            \end{equation} 
            beschreiben. Falls die Masse der Kristallelektronen als effektive Masse 

            \begin{equation}
                m_i^* = \frac{\hbar^2}{\left(\frac{\partial^2\varepsilon}{\partial k_i^2} \right)_{k=0}}
            \end{equation}
            angenommen wird, folgen die Kristallelektronen im approximierten Leitungsband 
            den gleichen Gleichungen wie freie Elektronen, ohne dass das Kristallpotential 
            extra berücksichtigt werden muss. Ebenso können die Gleichungen der klassischen Mechanik
            auf die Teilchen angewendet werden, falls kleine elektrische oder magnetische 
            Felder angelegt sind.
            Dabei muss h\"aufig die Richtungsabh\"angigkeit von $m_i^*$ nicht explizit 
            ber\"ucksichtigt werden, solange eine ausreichende Kristallsymmetrie in 
            alle Richtungen vorhanden ist. Dies ist bei GaAs der Fall.

        \subsection{Zirkulare Doppelbrechung}

            Bei der zirkularen Doppelbrechung sorgt ein Festkörper dafür, dass Licht, welches 
            ihn durchläuft, eine Rotation der Polarisation erfährt. Die einfallende linear polarisierte 
            Welle wird als eine Überlagerung aus einer rechts- und einer links-zirkular polarisierten 
            Welle betrachtet, die sich innerhalb des Festkörpers mit unterschiedlicher Phasengeschwindigkeit
            fortbewegen. Der Rotationswinkel der Polarisationsebene wird durch

            \begin{equation}
                \theta = \frac{\symup{L}}2(k_{\symup{R}}-k_{\symup{L}}) = 
                \frac{\symup{L}\omega}2
                \left(\frac{1}{v_{\symup{Ph}_{\symup{R}}}}-\frac{1}{v_{\symup{Ph}_{\symup{L}}}}\right)
                = \frac{\symup{L}\omega}{2\symup{c}}(\symup{n}_{\symup{R}}-\symup{n}_{\symup{L}})
            \end{equation}

            beschrieben, wobei es sich bei $L$ um die Länge des durchtretenen Materials handelt.
            Die Doppelbrechung wird mikroskopisch betrachtet durch die 
            Nebendiagonalelemente des Suszeptibilitätstensors verursacht. 
            Die Suszeptibilität verknüpft die Polarisation mit dem elektrischen Feld über die 
            Gleichung 

            \begin{equation}
                \vec{P}=\varepsilon_0\chi\vec{E}.
            \end{equation}

            Der Tensor doppelbrechender Materialien hat die Form

            \begin{equation}
                \begin{pmatrix*}[r]
                    \chi_{xx} & i\chi_{xy} & 0 \\
                    -i\chi_{xy} & \chi_{xx} & 0 \\
                    0 & 0 & \chi_{zz}
                \end{pmatrix*}.
                \label{eqn:tensor}
            \end{equation}

            Dadurch lässt sich der Winkel der Faraday-Rotation durch das Nebendiagonalelement 
            ausdrücken:

            \begin{equation}
                \theta \approx \frac{\symup{L}\omega}{2\symup{c}^2}v_{\symup{Ph}}\chi_{xy}.
            \end{equation}


        \subsection{Faraday-Effekt}

            Beim Faraday-Effekt verändert ein Material, an welches ein magnetisches Feld 
            in Strahlrichtung angelegt ist, die Polarisationsebene des durchtretenden Lichts. 
            Auch bei Materialien, die grundsätzlich einen diagonalen Suszeptibilitätstensor besitzen,
            treten durch das Magnetfeld Nebendiagonalelemente auf, die zu Doppelbrechung und somit 
            letztendlich zum Faraday-Effekt führen. Der Tensor hat die Gestalt aus Gleichung \eqref{eqn:tensor}.

            %\begin{equation}
            %    \begin{pmatrix*}[r]
            %        \chi_{xx} & i\chi_{xy} & 0\\
            %        -i\chi_{xy} & \chi_{xx} & 0\\
            %        0 & 0 & \chi_{zz}
            %    \end{pmatrix*}.
            %\end{equation}

            Explizit lässt sich das Matrixelement als 

            \begin{equation}
                \chi_{xy} = \frac{N e_0^3 \omega B}
                {\varepsilon_0\left(\left(-m\omega^2+K\right)^2-(e_0\omega B)^2\right)}
            \end{equation}

            darstellen, wobei $K$ eine Bindungskonstante des Elektrons ist und $N$ die Zahl der 
            Ladungsträger pro Volumeneinheit darstellt. Durch die Kenntnis des Matrixelements 
            kann der Winkel der Faraday-Rotation durch 

            \begin{equation}
                \theta = \frac{e_0^3}{2\varepsilon_0\symup{c}}\frac{\omega^2}
                {\left(-m\omega^2 +K\right)^2-(e_0\omega B)^2}\frac{NBL}{n}
            \end{equation}
            ausgedrückt werden. Falls die Messfrequenz weit unter der Resonanzfrequenz 
            gebundener Ladungsträger liegt, kann der Winkel als 

            \begin{equation}
                \theta \approx \frac{2\pi^2e_0^3\symup{c}}{\varepsilon_0}\frac1{m^2}
                \frac1{\lambda^2\omega_0^4}\frac{NBL}n
            \end{equation}

            geschrieben werden. Für freie Elektronen oder gebundene Elektronen in der Schreibweise der 
            effektiven Masse, d.h. f\"ur die quasifreien Elektronen
            im Leitungsband,  geht die Resonanzfrequenz gegen 0.
            Dies führt zu folgendem Ausdruck für den 
            Faraday-Winkel:

            \begin{equation}
                \theta \approx \frac{e_0^3}{8\pi^2\varepsilon_0\symup{c}^3}\frac1{m^2}\lambda^2\frac{NBL}n.
            \end{equation}
