\section{Diskussion}
\label{sec:Diskussion}

Die ermittelten Lande-Faktoren und ihre theoretischen Werte sind in Tabelle \ref{tab:lande} aufgelistet.

\begin{table}
  \centering
  \begin{tabular}{c c c}
    \toprule
     & $g_{ij}$ & $g_{\symup{theo}}$\\
    \midrule
        $\sigma_{\symup{rot}}$   &   $\SI{1.8(2)}{}$    & $\SI{1}{}$  \\
        $\sigma_{\symup{blau}}$  &   $\SI{1.9(3)}{}$    & $\SI{1.75}{}$  \\
        $\pi_{\symup{blau}}$     &   $\SI{0.70(13)}{}$  & $\SI{0.5}{}$  \\
    \bottomrule
  \end{tabular}
  \caption{Ermittelte Lande-Faktoren und theoretische Werte.}
  \label{tab:lande}
\end{table}

Nur der Faktor für die $\sigma$-Komponente der blauen Spektrallinie liegt in einem Fehlerintervall mit dem theoretischen Wert.
Der theoretische Wert $\SI{1.75}{}$ setzt sich dabei aus dem Mittelwert der Lande-Faktoren für $\Delta m= 0,\pm1$ zusammen, da die
aufgespalteten Linien nicht voneinander unterschieden werden können. Während der Wert für die rote Spektrallinie eine hohe Abweichung
zum theoretischen Wert hat, liegt die $\pi$-Komponente der blauen Spektrallinie fast in einem Fehlerintervall mit ihrem theoretischen Wert.
%Die hohe Abweichung bei der roten Spektrallinie, liegt zum Teil in der geringen Anzahl an messbaren Abständen zwischen den Maxima.

Ein Grund für die sichtbare Abweichung liegt in der Messung des Magnetfeldes. Diese ist sehr fehleranfällig
und kann zu ungenauen Werten führen. Weiterhin wurde keine Ungenauigkeit auf das Messen der Abstände zwischen den Maxima
betrachtet. 

Insgesamt konnte nur für die $\sigma$-Komponente der blauen Spektrallinie der theoretische Wert bestätigt werden.