\section{Diskussion}
\label{sec:Diskussion}

Die ermittelten Werte für den Untergrund betragen $\symup{U_{C}} = 4.592$ und
$\symup{U_{Fit}} = 1.43\,\pm\,0.18$. Damit ist der erste Wert mehr als dreimal so
groß wie der andere.

Die bestimmte Lebensdauer des Myons liegt bei $\symup{\tau} = (1.98\;\pm\;0.03)\,\symup{\mu s}$.
Der Theoriewert liegt dagegen bei $\symup{\tau_{theo}} = 2.197\,\symup{\mu s}$ \cite{pdg}, und ist
damit nicht im Fehlerintervall des ermittelten Wertes.\\
\ \\
\noindent
Die große Abweichung bei dem Untergrund könnte aus den Messwerten bei dem Plateau kommen, da
dort die lineare Regression nicht wirklich auf die Daten passt. Ein möglicher Grund für diese
hohen Abweichungen bei der Plateaumessung könnte ein systematischer Fehler in der Veränderung
der Verzögerungszeit sein.

Der Grund für die Abweichung der Lebensdauer könnte einer fehlerhaften Zuordnung der Signale
vom Vielkanalanalysator zugrunde liegen. Dieses Phänomen war schon bei der Kalibrierung sichtbar,
und könnte die Messdaten verfälschen.
