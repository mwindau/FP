\section{Diskussion}
\label{sec:Diskussion}

Die ermittelten Werte für den Untergrund betragen $\symup{U_{C}} = 3,676$ und
$\symup{U_{Fit}} = 1,26\,\pm\,0,19$. Damit ist der erste Wert mehr als dreimal so
groß wie der andere.

Die bestimmte Lebensdauer des Myons liegt bei $\symup{\tau} = (2,03\;\pm\;0.03)\,\symup{\mu s}$.
Der Theoriewert liegt dagegen bei $\symup{\tau_{theo}} = 2.197\,\symup{\mu s}$ \cite{PhysRevD.98.030001}, und ist
damit nicht im Fehlerintervall des ermittelten Wertes.\\
\ \\
\noindent
Eine mögliche Ursache für die zuvor erwähnten Ausreißer in der Signalanzahl, könnte in
der Reflexion der Signale im Versuchsaufbau liegen. Dadurch würde ein Stoppsignal durch
die Reflexion des Eingangssignal erfolgen, und die Messung zu früh beenden.

Ein Grund für die Abweichung der Lebensdauer könnte eine fehlerhaften Zuordnung der Signale
vom Vielkanalanalysator sein, so wie es zuvor diskutiert wurde. Zudem wurde auch nicht das Phänomen der myonischen Atome betrachtet,
indem ein stark abgebremstes Myon eine Bindung mit einem Atomkern eingeht, und daher erst später
zerfällt.
