\section{Diskussion}
\label{sec:Diskussion}

\raggedright

Bei der Stabilitätsmessung gibt es kaum Übereinstimmung zwischen den Messwerten und
den Erwartungen nach Abbildung \ref{fig:vorbereitung}. Während bei der planar-konkaven Konfiguration
zumindest eine abfallende Tendenz zu erkennen ist, lässt sich bei der konkav-konkaven Konfiguration
nur durch auslassen einiger Messdaten der Fit an die Werte anpassen.

In der Polarisationsmessung wird ein quadratischer Kosinus-Term mit konstanter Phase
als Fitfunktion verwendet. Der Grund dafür ist, dass der Laser bereits linear polarisiertes Licht
aussendet, welches durch einen Kosinus mit konstanter Phase beschrieben werden kann, und
der Strahl durch den Polarisationsfilter eine weitere Kosinuspotenz erhält.
Diese Funktion spiegelt sich ziemlich genau in den Messwerten wieder.

Die Grundmodenmessung lässt sich ziemlich genau mit einer Gaußverteilung fitten.
Bei der ersten Mode wird eine asymmetrische doppelte Gaußverteilung angenommen, da
durch die Breite des Drahtes, und den dadurch enstehenden Schatten, eine Seite gesenkt wird.

Die ermittelte Wellenlänge liegt bei $\SI{629(4)}{\nano\meter}$, und ist daher im
Bereich von rotem sichtbaren Licht. Der Literaturwert liegt mit $\SI{632.82}{\nano\meter}$ im
Fehlerbereich des ermittelten Wertes. Dabei ist jedoch zu bemerken, dass mit den zwei
Längen zwischen den Maxima zu wenig Messwerte vorhanden sind um eine sichere Aussage zu treffen.
