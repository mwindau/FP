\section{Messwerte}
\label{sec:Messwerte}


      \begin{table}
        \centering
        \begin{tabular}{c c c c}
          \toprule
          Abstand in cm & Intensität in µA\\
          \midrule
              50,0  &    25,2\\
              54,0  &    12,6\\
              56,9  &  25,2\\
              58,7  &  5,4\\
              59,6  &  33,0\\
              61,9  &  23,5\\
              63,5  &  6,3\\
              64,5  &  21,2\\
              67,4  &  10,2\\
              70,6  &  9,1\\
              73,3  &  2,6\\
              77,3  &  12,0\\
              79,7  &  10,5\\
              81,3  &  12,2\\
              82,4  &  10,3\\
              85,0  &  11,7\\
              85,6  &  11,6\\
              88,2  &  5,0\\
              89,4  &  8,3\\
              90,9  &  3,1\\
              92,6  &  6,2\\
              93,0  &  3,9\\
              97,6  &  2,3\\
              100,9 &  0,8\\
              103,5 &  1,0\\
          \bottomrule
        \end{tabular}
        \caption{Abstand und Intensität bei einem planaren und einem konkaven Spiegel.}
        \label{tab:planarkonkav}
      \end{table}

      \begin{table}
        \centering
        \begin{tabular}{c c c c}
          \toprule
          Abstand in cm & Intensität in µA\\
          \midrule
              50,5    & 38,5  \\
              54,8    & 31,5  \\
              58,4    & 29,5  \\
              64,5    & 29,0  \\
              67,7*   & 52,0* \\
              70,0*   & 44,0* \\
              73,5*   & 46,6* \\
              78,1*   & 37,0* \\
              82,4*   & 38,6* \\
              88,1*   & 29,0* \\
              91,0*   & 30,1* \\
              93,5*   & 29,4* \\
              96,5*   & 23,5* \\
              100,0*  & 16,0* \\
              104,0*  &  7,4* \\
              107,4*  &  7,2* \\
              115,1*  & 14,0* \\
              120,0*  & 15,0* \\
              124,5   & 48,0  \\
              130,0   & 62,0  \\
              137,2   & 19,0  \\
          \bottomrule
        \end{tabular}
        \caption{Abstand und Intensität bei zwei konkaven Spiegeln.}
        \label{tab:konkavkonkav}
      \end{table}

    \begin{table}
      \centering
      \begin{tabular}{c c c c}
        \toprule
        Winkel in ° & Intensität in µA\\
        \midrule
          0   &  4,10 \\
          10   &  3,30\\
          20   &  2,80\\
          30   &  2,00\\
          40   &  1,00\\
          50   &  0,41\\
          60   &  0,12\\
          70   &  0,05\\
          80   &  0,37\\
          90   &  0,93\\
          100   &  1,61\\
          110   &  3,80\\
          120   &  4,80\\
          130   &  3,80\\
          140   &  5,00\\
          150   &  4,20\\
          160   &  5,60\\
          170   &  4,20\\
          180   &  4,00\\
          190   &  2,60\\
          200   &  1,60\\
          210   &  1,40\\
          220   &  0,96\\
          230   &  0,34\\
          240   &  0,06\\
          250   &  0,04\\
          260   &  0,20\\
          270   &  0,52\\
          280   &  1,25\\
          290   &  1,47\\
          300   &  1,53\\
          310   &  2,00\\
          320   &  2,65\\
          330   &  2,90\\
          340   &  4,30\\
          350   &  4,55\\
          360   &  4,20\\
        \bottomrule
      \end{tabular}
      \caption{Winkel des Polarisationsfilters und Intensität.}
      \label{tab:polarisation}
    \end{table}



    \begin{table}
      \centering
      \begin{tabular}{c c c c}
        \toprule
        Position in mm & Intensität in nA\\
        \midrule
        -30  & 0,011\\
        -28  & 0,035\\
        -26  & 0,060\\
        -24  & 0,111\\
        -22  & 0,165\\
        -20  & 0,300\\
        -18  & 0,500\\
        -16  & 0,730\\
        -14  & 0,920\\
        -12  & 1,000\\
        -10  & 1,200\\
         -8  & 1,280\\
         -6  & 1,150\\
         -4  & 1,070\\
         -2  & 1,000\\
          0  & 0,900\\
          2  & 0,750\\
          4  & 0,650\\
          6  & 0,500\\
          8  & 0,400\\
         10  & 0,300\\
         12  & 0,210\\
         14  & 0,140\\
         16  & 0,077\\
         18  & 0,038\\
         20  & 0,017\\
         22  & 0,007\\
         24  & 0,003\\
         26  & 0,001\\
         28  & 0,002\\
         30  & 0,001\\
        \bottomrule
      \end{tabular}
      \caption{Position der Diode und gemessene Intensität für die Grundmode.}
      \label{tab:grundmode}
    \end{table}


    \begin{table}
      \centering
      \begin{tabular}{c c | c c}
        \toprule
        Position in mm & Intensität in µA & Position in mm & Intensität in µA\\
        \midrule
        -31  &  2,8  &    1  & 36,0\\
        -30  &  3,0  &    2  & 47,0\\
        -29  &  4,7  &    3  & 58,0\\
        -28  &  7,2  &    4  & 68,0\\
        -27  & 10,9  &    5  & 75,0\\
        -26  & 12,4  &    6  & 80,0\\
        -25  & 18,0  &    7  & 78,0\\
        -24  & 24,5  &    8  & 77,0\\
        -23  & 30,5  &    9  & 69,0\\
        -22  & 35,0  &   10  & 67,0\\
        -21  & 41,0  &   11  & 66,0\\
        -20  & 54,0  &   12  & 57,0\\
        -19  & 69,0  &   13  & 51,0\\
        -18  & 78,0  &   14  & 46,0\\
        -17  & 87,0  &   15  & 35,0\\
        -16  & 91,0  &   16  & 26,0\\
        -15  & 93,0  &   17  & 21,0\\
        -14  & 92,0  &   18  & 13,0\\
        -13  & 92,0  &   19  &  9,0\\
        -12  & 85,0  &   20  &  4,8\\
        -11  & 78,0  &   21  &  2,6\\
        -10  & 68,0  &   22  &  1,3\\
         -9  & 52,0  &   23  &  0,7\\
         -8  & 35,0  &   24  &  0,6\\
         -7  & 19,5  &   25  &  0,7\\
         -6  &  8,5  &   26  &  0,8\\
         -5  &  1,7  &   27  &  0,8\\
         -4  &  0,5  &   28  &  0,7\\
         -3  &  2,3  &   29  &  0,5\\
         -2  &  8,0  &   30  &  0,2\\
         -1  & 15,0  &   31  &  0,2\\
          0  & 25,0\\
        \bottomrule
      \end{tabular}
      \caption{Position der Diode und gemessene Intensität für die erste Mode.}
      \label{tab:erstemode}
    \end{table}
