\section{Theorie}
\label{sec:Theorie}


    Ein Laser, dessen Bezeichnung eine Abkürzung für Light Amplification by Stimulated Emission of Radiation ist,
    sendet Licht aus, welches monochromatisch, sehr intensiv und kohärent ist.
    Ein Helium-Neon-Laser enthält ein Gasgemisch, in dem das Verhältnis von Helium zu Neon 5:1 beträgt.
    Neon ist dabei das aktive Lasermedium, das für das Strahlungsspektrum des Lasers verantwortlich ist.
    Helium ist das sogenannte Pumpgas, das nötig ist, um eine Besetzungszahlinversion durchzuführen.
    Dies funktioniert vereinfacht betrachtet folgendermaßen:
    Die Neon-Atome besitzen einen Grundzustand und einen angeregten Zustand.
    Im thermischen Gleichgewicht liegen im Neon mehr Atome im Grundzustand als in einem angeregten Zustand vor.
    Die Helium-Atome werden von außen durch sogenanntes "Pumpen" kontinuierlich angeregt
    und geben bei Stoßprozessen ihre Energie an die Neon-Atome ab.
    Dadurch liegen am Ende mehr Neon-Atome im angeregten Zustand als im Grundzustand vor.
    Die Atome im Grundzustand können ein Photon absorbieren und bei entsprechend hoher Energie
    in den angeregten Zustand übergehen. Die Atome im angeregten Zustand können spontan in den Grundzustand zurückfallen
    und dabei Photonen emittieren.
    Bei der stimulierten Emission wird ein Atom im Grundzustand durch ein Photon gezielt angeregt, damit
    es in den höheren Zustand übergehen und daraus wieder ein Photon emittieren kann.
    Durch Besetzungszahlinversion wird die stimulierte Emission wahrscheinlicher als die spontane,
    sodass sich die Kohärenz des Lichtes signifikant verbessert.\\
    In der Realität besitzen Helium und Neon mehr als 2 mögliche Zustände.
    Das Helium wird durch äußere Energiezufuhr entweder auf sein $2^1S_0$-Niveau oder auf sein $2^3S_1$-Niveau
    gebracht. Diese haben ungefähr die gleiche Energie wie das $5s-$ und das $4s-$Niveau von Neon,
    sodass die Neon-Atome bei Stößen auf diese Niveaus gehoben werden können. Von dort haben die
    Atome verschiedene Möglichkeiten, Photonen abzugeben. Am häufigsten liegt der Übergang vom $5s-$ ins $3p-$Niveau
    vor, weshalb die Linie mit der Wellenlänge 632,8\,nm
    ~\cite{anleitungv61} am intensivsten ist. \\

    Neben Lasermedium und Pumpgas benötigt ein Laser einen Resonator, der zu einer Rückkopplung und somit
    zu einer Verstärkung des Lichts führt. Er wird durch zwei gegenüberliegende Spiegel realisiert, die
    dafür sorgen, dass das aktive Lasermedium von einem Großteil des Laserstrahls mehrfach durchlaufen wird.
    Dadurch, dass einer der Spiegel teildurchlässig ist, kann ein kleiner Teil des Strahls entweichen.
    Die Form der beiden Resonatorspiegel, welche gekrümmt oder planar sein können,
    hat einen Einfluss auf die Eigenschaften des Resonators.
    Der Resonatorparameter $g_i$ wird aus der Gleichung

    \begin{equation}
        g_i = 1 -\frac{L}{r_i}
    \end{equation}

    bestimmt, wobei $r_i$ die Radien der Spiegelkrümmung sind und $L$ die Resonatorlänge ist.
    Ein optisch stabiler Resonator liegt vor, wenn für die Resonatorparameter der beiden Spiegel gilt:

    \begin{equation}
        0 \leq g_1 \cdot g_2 < 1\,\,.
    \end{equation}

    Möglichst geringe Verluste durch die Spiegel können erzielt werden, indem sich beide Brennpunkte am gleichen
    Ort befinden.\\

    Im Resonator kommen longitudinale und transversale Moden vor, von denen für die hier vorgenommenen Untersuchungen aber nur
    die transversalen wichtig sind, die durch nicht optimale Versuchsbedingungen wie Spiegelunebenheiten hervorgerufen werden.
    $l$ ist die Knotenanzahl in x-Richtung und $q$ die Knotenanzahl in y-Richtung.
    Somit werden die transversalen Moden in der Form $\symup{TEM}_{lq}$ angegeben.
    TEM steht dabei für transverse electromagnetic mode.
    Kleinere Moden haben eine höhere Symmetrie und deshalb weniger Verluste, weshalb die Grundmode $\symup{TEM}_{00}$
    am besten darstellbar ist. Ihre Intensitätsverteilung ist gaußförmig.
