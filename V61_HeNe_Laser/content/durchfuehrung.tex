\section{Durchführung}
\label{sec:Durchführung}

    Im ersten Versuchsteil wird die Intensität des Lasers in Abhängigkeit von der Resonatorlänge bestimmt,
    um die Stabilitätsbedingung zu überprüfen.
    Dazu muss grundsätzlich links vom HeNe-Laser ein Spiegel angebracht werden und rechts vom Laser in
    etwa gleichem Abstand ein weiterer. Die Intensität wird für verschiedene Spiegelabstände mithilfe der
    Photodiode gemessen. Um eine hohe Messqualität zu erreichen, müssen beide Spiegel vorher einjustiert werden.
    Dazu wird zunächst der erste Spiegel gemeinsam mit einem Fadenkreuz in die optische Schiene gestellt und
    der Justierlaser eingeschaltet.
    Die Spiegelposition bezüglich der optischen Achse wird mithilfe von Justierschrauben so lange verändert,
    bis der Punkt des Justierlasers sich genau in der Mitte des Fadenkreuzes befindet.
    Der gleiche Vorgang wird für den zweiten Spiegel wiederholt.
    Nach der Justage sollte der HeNe-Laser sichtbar sein, und der Justierlaser ausgeschaltet werden. %wird der Justierlaser aus- und der HeNe-Laser angeschaltet.
    In der ersten Messung werden ein planarer und ein konkaver Spiegel mit Krümmungsradius 1400\,mm
    verwendet. Es werden 25 Messwerte aufgenommen bei Abständen zwischen 50 und 104\,cm.
    Die zweite Messung ermittelt den Zusammenhang bei zwei konkaven Spiegeln mit den Krümmungsradien 1400\,mm.
    Hier werden 22 Intensitäten für Abstände zwischen 50 und 140\,cm gemessen.\\
    Für die folgenden Messungen werden zwei konkave Spiegel verwendet.\\
    Um die Polarisation des Lichts zu messen, wird ein Polarisationsfilter vor die Photodiode geschraubt.
    Dieser wird in 10°-Schritten in einem Bereich von 0 bis 360° gedreht. Die zugehörigen Intensitäten werden gemessen. \\
    Als nächstes wird die TEM-Grundmode vermessen. Dazu wird eine Zerstreuungslinse vor die Photodiode geschraubt, sodass
    der Laserstrahl quasi vergrößert wird. Danach wird die Photodiode auf einer Schiene senkrecht zur optischen Achse
    bewegt, um die Intensität des Strahls in verschiedenen Bereichen zu erfassen. Insgesamt werden 31 Intensitäten auf einer
    Strecke von 60\,mm aufgenommen. \\
    Um die $\symup{TEM}_{01}$-Mode zu vermessen, wird ein 0,005\,mm dicker Wolframdraht zwischen dem Laser-Gehäuse und dem Spiegel
    auf der Seite der Photodiode
    plaziert und die gleiche Messung wie im vorherigen Teil durchgeführt, mit dem Unterschied, dass 63 Messwerte bestimmt werden. \\
    Im letzten Teil wird die Wellenlänge des Lasers ermittelt. Dazu wird vor die Photodiode ein Schirm und davor ein Gitter geschraubt,
    welches eine Dichte von 100 Stäben pro mm hat.
    Auf dem Schirm können dann Beugungsmaxima nullter und erster Ordnung identifiziert werden. Der Abstand
    vom Hauptmaximum zu den Maxima erster Ordnung sowie der Abstand vom Gitter zum Schirm werden gemessen.\\
