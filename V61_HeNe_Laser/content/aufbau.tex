
\section{Aufbau}

    Die Grundlage des Aufbaus bildet eine optische Schiene, an welche alle benötigten Versuchskomponenten
    angeschraubt werden können.
    Fest mit der Schiene verbunden sind der Justierlaser am linken Ende, der Helium-Neon-Laser in der Mitte
    und eine Photodiode am rechten Ende.
    Weitere Elemente, wie verschiedene Spiegel mit Durchmesser 12,7\,mm, ein Schirm mit Fadenkreuz,
    ein Polarisationsfilter, ein Draht oder ein optisches Gitter,
    können je nach Bedarf an beliebigen anderen Stellen der Schiene hinzugefügt werden.
    Der verwendete Helium-Neon-Laser ist 408\,mm lang und hat einen Durchmesser von 1,1\,mm.
    An jedem Laserrohrenden befindet sich ein Brewster-Fenster, durch welches parallel polarisiertes Licht
    beinahe verlustfrei gelangen kann.
